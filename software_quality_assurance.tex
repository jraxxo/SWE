%% nicht vergessen draft raus zu nehmen, um echte Bilder einzubinden und die Problem-Vierecke verschwinden zu lassen
\documentclass[11pt,a4paper,oneside,svgnames]{report}

\usepackage[utf8]{inputenc}
\usepackage[T1]{fontenc}
\usepackage{lmodern}
\usepackage{eurosym}
\usepackage[british]{babel}
\usepackage{float}

\usepackage{longtable}
\usepackage{ae}
\usepackage{hyperref}
\usepackage[table]{xcolor}
\usepackage{colortbl}
\usepackage{multirow}
\usepackage{tabularx}
\usepackage{graphicx}
\usepackage{tikz}
\usepackage{kpfonts}
\usepackage[explicit]{titlesec}
\usepackage[acronym,nonumberlist,style=tree]{glossaries}
\usepackage{amssymb}
\usepackage[left=3.65cm,right=3.65cm]{geometry}
\usepackage{listings}


%BEGIN Chapter Definition

\makeatletter
\def\thickhrulefill{\leavevmode \leaders \hrule height 1ex \hfill \kern \z@}
\def\@makechapterhead#1{%
  \vspace*{10\p@}%
  {\parindent \z@ \raggedleft \reset@font
            \scshape \@chapapp{} \thechapter
        \par\nobreak
        \interlinepenalty\@M
    \Huge \bfseries #1\par\nobreak
    %\vspace*{1\p@}%
    \hrulefill
    \par\nobreak
    \vskip 50\p@
  }}
\def\@makeschapterhead#1{%
  \vspace*{10\p@}%
  {\parindent \z@ \raggedleft \reset@font
            \scshape \vphantom{\@chapapp{} \thechapter}
        \par\nobreak
        \interlinepenalty\@M
    \Huge \bfseries #1\par\nobreak
    %\vspace*{1\p@}%
    \hrulefill
    \par\nobreak
    \vskip 50\p@
  }}

%END Chapter Definition

%BEGIN Title Definition

\makeatletter
\def\thickhrulefill{\leavevmode \leaders \hrule height 1pt\hfill \kern \z@}
\renewcommand{\maketitle}{\begin{titlepage}%
    \let\footnotesize\small
    \let\footnoterule\relax
    \parindent \z@
    \reset@font
    \null\vfil
    \begin{flushleft}
      \huge \@title
    \end{flushleft}
    \par
    \hrule height 4pt
    \par
    \begin{flushright}
      \LARGE \@author \par
    \end{flushright}
    \vskip 60\p@
    \vfil\null
  \end{titlepage}%
  \setcounter{footnote}{0}%
}

%END Title Definition

%Verschissener rotierter Text für verschissene Tabelle 14.1/2
\makeatletter
\newsavebox\zzz
\def\mystrut{%
\dimen@\wd\zzz
\divide\dimen@\thr@@
\advance\dimen@-\dp\@arstrutbox
\rule\z@\dimen@}

\def\rotatezzz{%
\rotatebox{90}{\rlap{\kern-\dp\@arstrutbox\usebox\zzz}}}
%END Verschissener rotierter Text

\makeatother
\title{Software Quality Assurance for Project ``BookExpress''}
\author{Marc A. Harnos\\ {mharnos@gmail.com} \and Joscha Rapp\\ {jraxxo@gmail.com} \and Christian Schulz\\ {crs.s@gmx.net}}
\author{Marc A. Harnos\\ Joscha Rapp\\ Christian Schulz}
\date{October 2012}

\definecolor{tableHead}{HTML}{5393B7}
\definecolor{tableEven}{HTML}{DAF1FF}
\definecolor{tableOdd}{HTML}{ADD0E5}
\definecolor{tableFoot}{HTML}{5393B7}

\definecolor{linkcolour}{rgb}{0,0.2,0.6}

\hypersetup{colorlinks,breaklinks,urlcolor=linkcolour,linkcolor=linkcolour}
\renewcommand{\arraystretch}{1.25}

\begin{document}
\maketitle
\tableofcontents

\chapter*{Document History}

\begin{table}[H]
\centering
\begin{tabular}{|l|l|l|l|}
\hline 
Editor(s) & Date & Purpose of Editing & Version \\ 
\hline 
Harnos, Rapp, Schulz & 2012-10-01 & Initial Document Creation & v0.01 \\ 
\hline
Harnos, Rapp & 2012-10-08 & Fick Das & v0.02 \\ 
\hline
Schulz & 2012-10-08 & Fick Das Nicht - Überstimmt & v0.03 \\ 
\hline
\end{tabular} 
\caption{Document History Table}
\label{tab:document-history}
\end{table}

\rowcolors{1}{tableEven}{tableOdd}
\chapter{Introduction}
The software quality assurance concerns itself - as the title already suggests - with the issue of guaranteeing a certain level of quality of the final product. However, the testing does not only occur when the product is finished, as continuous testing is critical to keep the code free from errors early on - the earlier an error gets noticed, the less damage will be done by it. If you're making the mistake of testing at the end of the development phase, chances are pretty high that you may have some fundamental errors in your code, which possibly take a rather long time to fix, as so much more code is directly or indirectly relying on it. If you discover such an error, say, a few hours after it has been coded into the product, you'll be able to fix it rather quickly and everybody else working on the project can adapt themselves to it and change their code accordingly, which may take maybe half an hour or so - if you're discovering the mistake 2 weeks later, the possibility is very high that many others have already written a lot of code that depends on your underlying fundamentals, so a lot of code will have to get either fixed or maybe even rewritten completely, which costs a lot of working hours - and money. This is why continuous quality assurance is so very important - it may seem trivial and tedious at first, but it's nothing compared to the scenario that was just described.

\chapter{Software Testing}
There are multiple ways to test a program or system to ensure that it is free of errors and functions properly. All of those fall in either one of the two big categories:
\begin{itemize}
\item \textbf{Static Testing} or
\item \textbf{Dynamic Testing}.
\end{itemize}

\section{Static Testing}
The big difference between these two is that static methods, such as reviews, walkthroughs or inspections do not require the tester to actually run the program at all. Instead, they use certain tools to analyse the program in a static manner, i.e. scanning the source code for certain abnormalities that could possibly produce errors. One of the big advantages of static software testing methods is that very complex systems that require very powerful hardware to run properly can be analysed and tested without having the aforementioned assets. However, there are always certain types of errors that are very hard to spot when you're not really running the system using test datasets.   
\section{Static Analysis}
Static Analysis is a very powerful static method for software analysis, and also the most common one. It was developed in the mid-70s by Michael Fagan at IBM. The main idea is that a round of qualified inspectors are getting together to discuss a piece of work that shall be integrated into the system in production. The meeting is moderated by a moderator who gets elected beforehand; usually that is the most experienced inspector. The inspection consists of 6 stages:
\begin{itemize}
\item \textbf{Planning:} The inspection is planned by the moderator.
\item \textbf{Overview meeting:} The author describes the background of the work product.
\item \textbf{Preparation:} Each inspector examines the work product to identify possible defects.
\item \textbf{Inspection meeting:} During this meeting the reader reads through the work product, part by part and the inspectors point out the defects for every part.
\item \textbf{Rework:} The author makes changes to the work product according to the action plans from the inspection meeting.
\item \textbf{Follow-up:} The changes by the author are checked to make sure everything is correct.
\end{itemize}
It is important to note that the stages Preparation, Inspection meeting and Rework may be iterated as long as the predefined exit criteria (posed by the moderator) is not met. 
The moderator was already mentioned a few times, however, every person taking part in the code inspection has a certain role to fulfil:
\begin{itemize}
\item \textbf{Author:} The creator of the inspected work product.
\item \textbf{Moderator:} He is leading and coordinating the inspection.
\item \textbf{Reader:} The person reading through the document being inspected; the other inspectors point out defects and errors during the reading.
\item \textbf{Recorder/Scribe:} He documents the defects and mistakes that are found during the inspection.
\item \textbf{Inspector:} Every person who examines the document and points out mistakes and defects.
\end{itemize}

\section{Dynamic Testing}
Dynamic testing is described by executing actual programmed code for given test cases, as in contrast to static testing which involves reviews, inspections and walkthroughs.

Dynamic testing can be performed even if the program itself is not finished yet to test parts of the software, especially individual modules and interfaces. Dynamic testing can be performed automatically on code compilation in form of, for example, unit tests.

There are several types of testing, approaches - one of them is the "box approach", which is separated into "white box"/"glass box" and "black box" testing; there is also a mixture of those approaches, the "grey box" testing.

\subsection{White Box / Glass Box Testing}
White box testing tests internal program functionality and structures, rather than the user functionality. For white box testing, the internal system perspective as well as the programming skills are used to design the test cases - the tester has to know the inner workings of the program to develop those test cases.

Common tests with white box testing include, API testing, fault injection and mutation testing.

A disadvantage of white box testing is, that it does not detect missing functionality of the program, but only implemented errors - so missing design specifications might not be detected.

\subsection{Black Box Testing}
As opposed to the white/glass box testing, in the black box testing the tester does not know anything about the inner workings of the software, the software is seen as a "black box". The tester knows what the software has to do, but not how it accomplishes the results.

Common tests performed with black box testing include, state transition tables, decision table testing, use case testing and specification-based testing.

\paragraph{Specification-based testing}
Specification-based testing takes the requirements of the software design and tests the software functionality for those requirements. The testing consists of previously developed test cases which should perform a certain output for a given input, an expected behaviour is tested. This test can be built with the requirements specifications.

\subsection{Testing levels}
\subsubsection{Unit Testing}
Unit testing, or module testing, test the proper functionality of parts of code, like modules, functions and interfaces. This type of testing is usually written before, or at the same time as the code is developed, by the developers themselves to test if everything is working as expected; as described in the design.

This kind of test will never turn out positive if the function or interface is not implemented by design specifications. Unit testing is a kind of white box testing.

\subsubsection{Integration Testing}
Integration testing tests if interfaces work together according to software design. Such tests can detect interfaces not working properly in the whole scheme of the application, so finding them and fixing the issue is made possible/faster.

Integration tests are used to test if modules/components integrate into interfaces and make sure that the system works as one unit in the end.

\subsubsection{System Testing}
System testing tests the whole system (as one unit) for compliance with the design specifications. As opposed to unit and integration testing, system testing is a kind of black box testing, so no inner workings of the system have to be known to the tester - the tester does only have to test the system as a whole against the specifications.

System testing is in general only performed after the integration testing passes successfully.

\subsubsection{Acceptance Testing}
Acceptance testing is, much like system testing, a black box testing method for testing the whole software against agreed system design/the specification.

The main difference between system testing is, that, as opposed to system testing, the acceptance testing is performed by the client/user - also called user acceptance testing (UAT) or end-user testing.

With a successful acceptance test, the software is delivered to the client as a finished product and testing usually ends here.

\chapter{Test Cases For Black Box Testing}
\section{Testing of actions carried out by the employees of BookExpress}
\subsection{/PF00/ Register Book Shop (Client) or Publisher}
To determine if the function works properly, we will register a new Book Store and a new publisher. 
We will use the following test data:
 
\begin{center}
\label{client1}
\textbf{Book Store}
\begin{itemize}
\item{ClientPIN: 012345}
\item{Account Type: Book Store}
\item{Address: 123 Example Avenue, 456 GreatTown}
\item{Correspondent: John Doe}
\item{Login: 012345 321 password}
\end{itemize}
\label{publisher1}
\textbf{Publisher}
\begin{itemize}
\item{ClientPIN: 045678}
\item{Account Type: Publisher}
\item{Address: 123 Main Street, 456 Jones Town}
\item{Correspondent: Malory Miller}
\item{Login: 045678 123 aterriblesecret}
\end{itemize}
\end{center}
\subsection{/PF01/ Edit client information}
Now that we have a few Accounts, we want the client data to be correct and up-to-date, so we will test the function with the following updated information:
\begin{itemize}
\item{ClientPIN: 012345}
\item{Address: 321 Example Street, 456 GreatTown}
\item{Correspondent: Jane Doe}
\item{Login: 012345 321 password}
\end{itemize}
\subsection{/PF02/ Delete Account}
Now, we're dealing with the unfortunate Situtation that a client has decided to close his business. As we do not want our database to get clogged up with outdated and unnecessary information, we'll delete the client with the PIN 012345 from our database.
\subsection{/PF10/ Client places Order via Phone}
One of the most important functions of the system is, of course, the ordering process. As there are two ways of doing this, we'll have to test both of them. We will start by testing the 'old-fashioned' way of ordering books via a simple phone call. For this test, the following data shall be used:
\begin{itemize}
\label{testbook}
\item{Title: The big test Book}
\item{ISBN: 100101101001101101}
\item{Author: Arthur Miller}
\item{Publisher: Test Books Ltd.}
\item{Price: 30,00}
\item{Amount: 100}
\end{itemize}
\subsection{/PF11/ Validate Order}
\label{automatic1}
As this product function will be carried out automatically by the system after an order has been submitted, we do not require specific test data for it. It will get invoked when we submit our test data from /PF/10.
\subsection{/PF12/ Stock Update}
This product function can be invoked automatically (when an order is processed by the system) or manually (when a BookExpress employee decides to initiate it). 
We shall test this function specifically by invoking it manually with the test data we introduced earlier at \ref{testbook}.
\subsection{/PF13/ Forward order to logistics}
See above at \ref{automatic}.
\subsection{/PF14/ Deliver Order}
See above at \ref{automatic}.
\subsection{/PF15/ Confirm Delivery}
The confirmation is also being carried out automatically when an order is processed completely. Therefore, see above at \ref{automatic}.
\subsection{/PF16/ BookExpress Order}
As the publishers are the only sources that BookExpress gets its books from, we will need to order books from them that are not available in our stock anymore.
This functionality shall be tested using the test data from above(\ref{testbook}), as this has to happen: we are starting with a stock of zero books in our hyptothetical scenario and the customer orders this exact book, so we'll have to order it directly from the publisher before being able to ship it to our client.
Because we want to ensure that we'll have the book in stock if they order it again, we will order twice the requested amount
\subsection{/PF20/ Create an invoice}
See above at \ref{automatic}.
\subsection{/PF21/ Send invoice}
See above at \ref{automatic}.
\subsection{/PF30/ Backup}
The Backup process is automatically started every 24 hours at 00:00. To see if the process runs properly, the system will either need to run for a whole day and night or we could just manipulate the system time of the underlying Operating System in order to save time if needed.
\section{Book Stores and Publishers}
\subsection{/PF03 Login}
To ensure that clients and publishers are able to successfully access the system by logging in, we'll use the test data from \ref{client1} and \ref{publisher1} to test this functionality. 
\subsection{/PF04/ Update account information}
If a client or a publisher has, for example, physically moved his location of business or hired a new correspondent, they may want to update their contact data themselves as it's less complicated than notifying a BookExpress employee about the changes. To test this functionality, we will log in with \ref{publisher1}(the publisher test account) and update his information as follows:
\begin{itemize}
\item{ClientPIN: 045678}
\item{Account Type: Publisher}
\item{Address: 123 Main Street, 456 Jones Town}
\item{Correspondent: Dean Winchester}
\item{Login: 045678 123 waspurged}
\end{itemize}
\subsection{/PF05/ Logout}
This test will be carried out after submitting the order from \ref{client1}.
\section{Book Stores}   
\subsection{/PF40/ Search for a book}
This function is needed by the client to add specific books to his order. This way, he can check if a book can be delivered by us at all before ordering and possibly receiving a negative response which would result in wasted time and a lot of frustration on the side of the client. To test this functionality, we shall log in as \ref{client1} and search for "test book". The result should be \ref{testbook}.
\subsection{/PF17/ Book Store Order}
\label{orderfinal}
This function is essentially the same as \ref{orderphone}, but it's carried out using the web interface instead of calling a BookExpress employee. In order to test it, we will cancel our first order using the cancel order function (\ref{cancel}) and order it again, this time using the web interface.
\subsection{/PF18/ Change Order}
In case a client has made a mistake in an order and it's been already submitted, he is able to change the order if it isn't shipped yet. We will take the order from \ref{client1} and change the ordered amount of books to 15.
\subsection{/PF19/ Cancel Order}
\label{cancel}
If a client has made an Order and he comes to the conclusion that he won't be needing it anymore, he may cancel the order if it isn't shipped yet. We will cancel the first order (\ref{orderphone}) from our test client(\ref{client1}). This way, we can test both the cancel function as well as the second order function.
\subsection{/PF50/ Tracking System}
The tracking system is very important as it enables the client to make relatively precise predictions concerning the delivery date and time of his order. This is achieved by a GPS tracking system in each of our trucks. To get the current status or position of his order, the client (\ref{client1}) will log into the system and check the tracking status of his test order from \ref{orderfinal}. Depending on the time that has passed, he should get a valid status like "Processing".
\section{Publishers}
\subsection{/PF60/ Inventory Maintenance}
If a publisher choses to discontinue a certain book or add a new book to his portfolio, he can change the list of available books that we can order from by himself using the web interface. First, before the test order from the client is received, we will log in as the publisher (\ref{publisher1}) and add the test book from \ref{testbook} to our inventory.  After we ordered and "received" the test book from \ref{testbook} as the client, we will log in as \ref{publisher1} and delete the test book (\ref{testbook}) from the inventory again.

\chapter{Quality model /\\FURPS}
The quality factors described by McCall and his colleagues represent one of a number suggested "check lists" for software quality. Hewlett-Packard developed a set of software quality factors that has been given the acronym FURPS - functionality, usability, reliability, performance, and supportability. The FURPS quality factors draw liberally from earlier work, defining the following attributes for each of the five major factors\footnote{http://www.1000sourcecodes.com/2012/05/software-engineering-furps.html}

This Quality Model and FURPS will evaluate Team 10 on the previously mentioned points.

\section{Functionality}
In this section the functionality of the software of Team 10 will be evaluated, including if everything in the specification sheet is met and if any additional functionality is incorporated in the requirements specification.

\subsection{Feature Sets}
Clients and publishers are able to work with "Bookcelerator" over a web interface, which they log in after authenticating with a gls{vpn} in "BookExpress". Employees of "BookExpress" work with a stand alone application, which is accessible only in their private network/Intranet.

BookExpress employees can not only make maintenance work with the application but also work with publishers or book shops which can not, or are not willing to, work with the web interface - the employees can make and work with new order.

\subsection{Security}
"Bookcelerator" has an advanced backup system which ensures consistent and secure data management and data storage. Security is a mayor concern to "BookExpress", so every connection in the system is secured by modern encryption technologies like SSL. "Bookcelerator" over the web interface is accessible through a gls{vpn} which not only ensures more secure data transferral but also makes it nearly impossible to be compromised through an attacker from the internet - a "man in the middle" attack could be only possible through an attacker in the Intranet of "BookExpress" itself.

\section{Usability}
In this section we will not only discuss ease of use and learnability but also the efficiency and elegance of "Bookcelerator".\footnote{http://www.useit.com/alertbox/20030825.html}

\subsection{Human Factors}
Human factors describe the ease of use and learnability of "Bookcelerator". The software is especially easy to use due to smart and appropriate usage of an advanced hinting system and error message system. Also "Boocelerator" displays drop downs with auto completion, where appropriate.

\subsection{Aesthetics}
The user interface is held simple, has much white space and is intuitively usable, with a steep learning curve.

\subsection{Documentation}
The "Bookcelerator" documentation is designed well with a non-technical user kept in mind, while writing it. The steps are described well and the language used is held simple and clear. The documentation is, while kept simple and clear, very detailed.


\section{Reliability}
Reliability is a mayor concern for "BookExpress", so the "Bookcelerator" application is designed with as little as possible down time in mind. An advanced backup system is also being used to speed up the recovery of the system in case of failure.

\subsection{Recoverability}
Due to the advanced backup system of "Bookcelecrator" on different dedicated backup servers recoverability is held high and the system can be performed in a very short time after a failure. Backups are performed very frequently so data losses are minimised or in some cases not present at all.

\subsection{Accuracy}
Search results are performed very accurate minimising data overhead and making searches for books as precise as possible. The accuracy is very important, small errors in ISBN numbers can create duplicate entries in the system and compromise data consistency - therefore this kind of data, and in general input data, is checked on multiple layers to ensure accurate results and consistent data.

\section{Performance}
Performance of software is very important; slow applications make users unhappy and increasingly more irritated. Also the delivery of books has to have a high-performance as well.

\subsection{Speed and Efficiency}
To ensure high performance "BookExpress" from team 10 employs IBM System~z mainframes which are extraordinary fast and reliable.



\chapter{Appendices}
\printglossaries\addcontentsline{toc}{section}{Glossary}\addcontentsline{toc}{section}{Acronyms}

\listoffigures\addcontentsline{toc}{section}{List of figures}
\listoftables\addcontentsline{toc}{section}{List of tables}

\end{document}