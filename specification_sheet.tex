\documentclass[a4paper]{article}
\author{Christian Schulz, Marc A. Harnos, Joscha Rapp}

\begin{document}
\title{BookExpress - Specification Sheet}
\date{October 2012}
\maketitle
\tableofcontents

\section{Purpose/Aims of the Product}
The goal of the project is to create a software that will drastically reduce the amount of work the employees of "BookExpress" spend on logistics and administration, thus speeding up the processes and making sure that every customer gets his or her orders as soon as possible, within the specified timeframe (24 hours). The current IT system is not as flexible or comfortable as it should be - as a result the business processes, such as maintaining a stock database that is consistent as well as up-to-date, are overly complex and time-consuming. The user interface has to be intuitive and performant, relieving the employees of a lot of administrative work and quite possibly reducing the time that new employees need to become familiar with it. By establishing a direct connection to the customers and publishers, all processes can be handled faster. This new IT system consists of Software as well as Hardware. 
\\
\section{Usage of the Product}
The new IT system will be used by employees of BookExpress as well as the customers and publishers themselves. With the new system, customers will be able to submit orders automatically, depending on their demands, e.g. if their stock is below a certain threshold or they need to get a book for a final customer. The publisher on the other hand can push updates in their stock in realtime - as soon as a new book is available, it can be ordered from BookExpress. Furthermore, the new user experience will be much more modern and streamlined - instead of the old text-based system, all users will be able to use a modern, web-based interface which can even be accessed on mobile devices such as smartphones and tablets, thus increasing the flexibility of employees and customers even more.
\\	
\section{Product Overview}
OVERVIEW - what else?
\\
\section{Product Functions}
hahahahahahaha. as if.
\\
\section{Product Data}
here be data
\\
\section{Product Performance}
it's fucking fast, too

\end{document}
